% !TeX root = main.tex
% !TeX spellcheck = en_US

%Welche Schutzziele werden bei einer möglichen Ausnutzung der Schwachstelle verletzt?
%Wurden Designprinzipien verletzt und wenn ja, welche?
%Wurden Fehler bei der Implementierung gemacht und wenn ja, welche?
%Hätte der Fehler durch Tests entdeckt werden können und wenn ja, durch welche
%Testverfahren?
%Wie wurde das Risiko der Schwachstelle bewertet (CVSS-Rating) und warum?
%Welche Umstände oder welches Umfeld erhöhen oder erniedrigen das Risiko für die
%Verwender des betroffenen Produktes, Dienstes oder der Komponente?
%Kann die Schwachstelle oder ihre Ausnutzung im Betrieb festgestellt oder mitigiert werden?
%Wenn ja, wie?
%Wie hätte die Schwachstelle vermieden werden können (besseres Design, Implementierung,
%Test, Betrieb)?
%Sonstige Kommentare und Beurteilungen zur Schwachstelle

\section{Theoretical Foundations}

\subsection{SSL-VPN}

Um die vorliegende Sicherheitslücke besser zu verstehen und einordnen zu können, muss zunächst geklärt werden was überhaupt SSL-VPN ist und wie es funktioniert. 

In erster Linie handelt es sich bei eiem SSL-VPN um ein normales VPN welches in zweiter Linie um SSL ergänzt wird. 

\subsection{Pulse Connect Secure}

\section{Description of the Vulnerability}
\label{sec:description}
The CVE-2019-11510 is a critical vulnerability that allows attackers to get arbitrary file reading access after they have send a special URI.\autocite{NVDCVE:online}\\
This vulnarability is part of the CWE-22 class which is associated with "Path Traversal". That means that by explicitly specifying an abnormal path that is "[...] intended to identify a file or directory [...]" \autocite{CWE22-Definition:online} it is possible to gain access to that file or directory without owning the required rights. This is made possible because the software uses an externally specified path and due to the way the software proceeds with that path. The effect is that the software resolves the given paths to files or folders that lie outside the resitriced directory.\autocite{CWE22-Definition:online}

\section{Path traversal}
\label{sec:path-traversal}
Allthough path traversal was already mentioned in the previous chapter, one must take a closer look at path traversal in order to fully understand how it works and how it can be avoided. This will be done during the course of this chapter.

\section{How can this vulnarability be exploited?}
\label{sec:exploitation}
As already mentioned in \ref{sec:description}, in order to exploit the vulnerability the attacker must send a request (e.g. via HTTP) to the target server that contains a path sequence used for path traversal (see chapter \ref{sec:path-traversal}) and a special URI for the file that the attacker want to gain access to.\autocite{Tenable2:online}\\
"When a user logs into the admin interface of the VPN [...]"\autocite{Tenable2:online}, the password is stored as plain-text within a MDB file (Microsoft Access Database). The corresponding file can be found at \textit{/data/runtime/mtmp/lmdb/dataa/data.mdb}. Since the attacker already has arbitrary access to all files of the system he can easyly obtain the admin password. With this information the attacker could perform further attacks, e.g. exploiting the CVE-2019-11508, which allows an attacker to upload harmful files while using the credentials he obtained since the credentials actually belong to an authanticated user.\autocite{Tenable2:online}

\section{Example usage}
\label{sec:example}

\section{Affected}
\label{sec:affected}
All versions from between 8.2 to 8.2R12.1, 8.3 to 8.3R7.1 and 9.0 to 9.0R3.4 are affected.\autocite{NVDCVE:online}

\section{Solution}
Since this vulnerability is caused by the unintended behaviour of Pulse Connect Secure while resolving provided paths and therefore the internal implementation of this software, the vulnerability has been removed by a software patch provided by Pulse Secure.